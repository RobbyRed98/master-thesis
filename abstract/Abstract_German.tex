Graphdatenbanken erfreuen sich in den letzten Jahren einer steigenden Beliebtheit. Diese Datenbanken eignen sich um Daten innerhalb einer Graph-Struktur zu hinterlegen und abzufragen. Doch trotz des Fortschritts der in den letzten Jahren im Graph-Umfeld erzielt wurden, weisen relationale Datenbanksysteme in den meisten Workloads noch immer eine bessere Performance und Speichernutzung auf als Graphdatenbanken.

Im Jahr 2020 veröffentlichte IBM DB2Graph, eine Graph-Erweiterung für das relationale Datenbanksystem DB2. DB2Graph versucht hierbei die hohe Performance und geringere Speichernutzung von relationalen Datenbanksystemen sowie die einfache Abfrage von Graph-Strukturen zu vereinen. Das Ziel der Forschung der Arbeit ist es, zu analysieren, für welche Use-Cases sich DB2Graph eignet. Dafür wird untersucht, inwiefern sich DB2Graph und Graphdatenbank voneinander unterscheiden, wie sich die Performance und Speichernutzung von DB2Graph im Vergleich zu relationalen und Graph-basierten Datenbanksystemen verhält, mit welchem Aufwand der Einsatz und Betrieb von DB2Graph verbunden ist und ob sich DB2Graph auch anpassen oder erweitern lässt. Abschließende  Untersuchungen im DB2Graph-Umfeld befassen sich mit der möglichen Zukunft von DB2Graph.