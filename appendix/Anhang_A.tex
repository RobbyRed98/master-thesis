\chapter{CD-ROM}

Auf der beiliegenden CD-ROM befinden sich neben der Arbeit und allen Internetquellen zusätzlich:
\begin{itemize}
    \item der Linkbench Quellcode,
    \item die Konfigurationen von Linkbench und der Datenbanksysteme und
    \item alle Aufzeichnungen, Logs und Ergebnisstatistiken, die bei den Messungen erstellt wurden. 
\end{itemize}

Der Quellcode, der im Rahmen dieser Arbeit implementiert wurde, sowie die Konfigurationen der Datenbanksysteme und von Linkbench werden dabei als Teil des digitalen Anhangs der Arbeit offengelegt, um eine Reproduktion der Messungen zu ermöglichen. 

Darüber hinaus werden auch alle bei den Messungen erzeugten Logs und Ergebnisstatistiken beigelegt, um die im Rahmen dieser Arbeit erzielten Ergebnisse und Informationen für zukünftige Forschungen oder Vergleiche von Datenbanksystemen zur Verfügung zu stellen. 

Der Quellcode für Linkbench kann dabei im Ordner \texttt{linkbench} auf der CD-ROM eingesehen werden. Die Konfigurationen von Linkbench und der jeweiligen Datenbanksysteme hingegen befinden sich gemeinsam mit den Ergebnisstatistiken und Linkbench- oder NMON-Logs im Verzeichnis \texttt{Messungen}.