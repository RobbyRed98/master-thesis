\chapter{Einleitung}
\label{einleitung}

In den letzten Jahren haben Graphdatenbanksystemen einen großen Zuwachs an Popularität erfahren \cite{db_engines_ranking_july}. Dies hat auch bei den etablierten Herstellern von relationalen Datenbanksystemen das Interesse daran geweckt, ebenfalls Produkte basierend auf dem Graphmodell zu entwickeln und zu veröffentlichen. Bei einem dieser etablierten Hersteller handelt es sich hierbei um IBM. IBM hat anders als andere Datenbankhersteller wie Oracle allerdings nicht mit der Entwicklung einer eigenen Graphdatenbank begonnen. Stattdessen hat IBM mit Db2 Graph eine Art Erweiterung für sein relationale Datenbanksystem Db2 entwickelt, welches dazu in der Lage ist lesende Graphanfragen in SQL-Anweisungen umzuwandeln, die von Db2 verarbeitet werden können \cite{sigmod_tian}. 

Über welche Performance Db2 Graph als Grapherweiterung verfügt, wurde dabei bereits in \cite{sigmod_tian} untersucht. Dabei wurde in \cite{sigmod_tian} die Performance von Db2 Graph gemessen und mit der von JanusGraph und einem weiteren anonymisierten nativen Graphdatenbanksystem verglichen. Wie sich die Performance von Db2 Graph im Vergleich zu anderen Branchengrößen wie Neo4j einordnet, ist jedoch bisher unerforscht. 

Ziel dieser Arbeit ist es, eine Performance-Analyse für Db2 Graph und Neo4j durchzuführen und die erzielte Performance der beiden miteinander zu vergleichen.

Dafür sollen die Messungen mit dem Benchmark Linkbench in \cite{sigmod_tian} mit Db2 Graph und Neo4j reproduziert werden. Die dabei erzielten Ergebnisse sollen anschließend ausgewertet und miteinander verglichen werden, sodass die Performance von Db2 Graph gegenüber Neo4j eingeordnet werden kann.

Die Reproduktion der Ergebnisse und Messungen aus \cite{sigmod_tian} wird dabei in dieser Arbeit angestrebt, da in ihr einen klaren Rahmen für die Performance-Analyse von Graphdatenbanksystemen definiert wird. So beschreibt \cite{sigmod_tian} bereits mit Latenz und Durchsatz Metriken, die sich zur Einordnung und Bewertung der Graphdatenbanksysteme eignen. Sie belegt darüber hinaus auch, dass sich Linkbench als Benchmark zum Messen der Performance von Db2 Graph eignet. Schließlich präsentiert sie Ergebnisse, die von Linkbench für Db2 Graph ermittelt wurden. 

Im Verlauf dieser Arbeit wird die Performance von Db2 Graph als eine Art hybrides Graphdatenbanksystem und Neo4j als natives Graphdatenbanksystem analysiert und miteinander verglichen. 

Zu Beginn werden dabei das Graphmodell, die für die Arbeit relevanten Abfragesprachen, die Funktionsweise von Db2 Graph und Linkbench genauer erläutert. Auf Db2 Graph wird dabei detailliert beschrieben, um den Leser dieser Arbeit einen Einblick in die Funktionsweise dieser Technologie zu geben und wichtige Aspekte und Einschränkungen zu erörtern, die sich auch auf die Performance-Analyse der Grapherweiterung auswirken.

Im Anschluss daran wird das Vorgehen bei der Performance-Analyse der Datenbanksysteme erläutert. Dabei wird erklärt, weshalb die eigentlich angestrebte Reproduktion der Messungen und Ergebnisse aus \cite{sigmod_tian} mit Db2 Graph und Neo4j im Rahmen der Arbeit nicht möglich ist. Des Weiteren wird darin auf das neue Vorgehen bei der Performance-Analyse eingegangen. 

Als Nächstes wird auf die Methode und Organisation der Messungen eingegangen, die für die Performance-Analyse vorgenommen werden. Im Zuge dessen werden unter anderem die Messparameter, Konfiguration der Datenbanksysteme und der Aufbau der Performance-Analyse genauer erörtert. So wird im Detail auf bestimmte Konfigurationsparameter bei den Messungen eingegangen und welche folgen sie für die Messungen haben. Es wird erläutert, welche Rolle die Datensatzgröße und -verteilung bei den Messungen spielen. Darüber hinaus werden die Metriken zur Erfassung der Performance der Datenbanksysteme besprochen. Außerdem wird auch auf die Umgebung und Ressourcen eingegangen, die bei der Durchführung der Messungen zur Verfügung stehen. Abschließend wird die Organisation der Messungen in Messreihen erörtert und der Ablauf der Messungen beschrieben.  

Im nächsten Schritt werden die für die Performance-Analyse der Datenbanksysteme notwendigen Implementierungsarbeiten aufgeführt. Dazu gehört die Wahl der Code-Basis für Linkbench sowie die daran vorgenommen Anpassungen und die Umsetzung der Adapter zur Anbindung der Datenbanksysteme an den Benchmark. 

Anschließend werden die bei den Messungen erzielten Ergebnisse präsentiert und zum Schluss werden die Ergebnisse analysiert und die Performance von Db2 Graph und Neo4j miteinander verglichen. Für den Vergleich wird hierbei in erster Linie ein Performance-Faktor ermittelt. Dieser gibt dabei an, wie hoch die Performance eines Datenbanksystems gemessen an dem System ist, dass die niedrigste Performance aufweist. Darüber hinaus wird auch darauf eingegangen, welchen Einfluss die Größe der Ergebnismenge und der Datensätze auf die Performance von Db2 Graph und Neo4j hat. Des Weiteren wird versucht, den Ursprung eines nicht erwarteten Phänomens bei den Messergebnissen zu klären. Dafür wird unter anderem auch die aufgezeichnete CPU- und IO-Auslastung der Datenbanksysteme bei den Messungen analysiert.