\section{Abfragesprachen}
In diesem Abschnitt wird auf die für die Arbeit relevanten Abfragesprachen:
\begin{itemize}
    \item SQL,
    \item Gremlin und
    \item Cypher eingegangen.
\end{itemize}
Ziel dieses Abschnittes ist es, eine kleine Übersicht über die Abfragesprachen zu geben, die im Kontext dieser Arbeit eine besondere Rolle spielen. Abschließend werden die wichtigsten Punkte des Abschnitts nochmal kurz zusammengefasst.

\subsection{SQL}
Bei SQL handelt es sich um eine Abfragesprache, die im Feld der relationalen Datenbanksysteme weit verbreitet ist \cite{sql_history}. Sie wurde von Donald D. Chamberlin und Raymond F. Boyce im Rahmen des \textit{System R} Projekts entwickelt \cite{sql_history}. SQL wurde dabei mit der Motivation entworfen, eine einfache Abfragesprache für relationale Daten zu entwickeln \cite{sql_history}. 

SQL selbst stellt hierbei eine deklarative Abfragesprache dar \cite{sql_history}. Der grundlegende Aufbau der Sprache und die Syntax können dabei \autoref{src:sql_example} entnommen werden. So ist in \autoref{src:sql_example} klar erkennbar, dass sich alle Operationen auf Tabellen und Spalten beziehen. Die Sprache arbeitet also -- wie erwartet -- in den Strukturen des relationalen Modells.

\begin{lstlisting}[caption={Beispiel SQL-Queries},language=SQL,label=src:sql_example]
/* Tabelle erstellen */
CREATE TABLE Autos (
    Fahrzeugnummer VARCHAR(50), 
    Marke VARCHAR(10), 
    Modell VARCHAR(10), 
    Baujahr INT,
    PRIMARY KEY(Fahrzeugnummer)
);

/* Daten in Tabelle schreiben */
INSERT INTO Autos 
(Fahrzeugnummer, Marke, Modell, Baujahr) 
("FZ-123456789", "Toyota", "Starlet", 1997);

/* Daten Abfragen */
SELECT Marke, Model, Baujahr FROM Autos 
WHERE Fahrzeugnummer = "FZ-123456789";

/* Daten Loeschen */
DELETE FROM Autos WHERE Fahrzeugnummer = "FZ-123456789";
\end{lstlisting}

Des Weiteren muss darauf hingewiesen werden, dass SQL als Sprache standardisiert wurde \cite{sql_history}. Allerdings gibt es heute trotz des Standards weiterhin sogenannte SQL-Dialekte \cite{sql_2017}. So können sich einige SQL Sprachelemente je nach Datenbanksystem oder Hersteller weiterhin unterscheiden \cite{sql_2017}. 

\subsection{Gremlin}

Bei Gremlin handelt es sich um eine Abfragesprache für  Graphdatenbanksysteme \cite{tinkerpop_2020}. Die Sprache steht dabei in direkter Verbindung mit dem TinkerPop-Projekt  \cite{tinkerpop_2020}. Zu den Datenbanksystemen, die die Abfragesprache nutzen, gehören neben Db2 Graph, das im Rahmen der Arbeit eine wichtige Rolle spielt, auch:
\begin{itemize}
    \item JanusGraph \cite{janusgraph_2020},
    \item Amazon Neptune \cite{neptune_2021} und 
    \item CosmosDB von Microsoft \cite{cosmosdb_2021}.
\end{itemize}

Für die Interaktion mit den Graphen bietet Gremlin dabei zwei verschiedene Abfragestile \cite{gremlin_paper}. So ist es einerseits möglich, imperativ einen sogenannten Graph-Traversal durchzuführen, um mit dem Graph zu interagieren \cite{gremlin_paper}. Anderseits ist es allerdings auch möglich, einen deklarativen Pattern-Matching-Stil zu wählen \cite{gremlin_paper}. Im Rahmen dieser Arbeit wurde bei Gremlin-Queries immer der imperative Graph-Traversal basierte Ansatz gewählt \cite{gremlin_paper}. Daher werden als Beispiel in \autoref{src:gremlin_example} lediglich imperative Queries aufgeführt. 

\begin{lstlisting}[caption={Beispiel Gremlin-Queries},language=JAVA,label=src:gremlin_example]
/* Daten in den Graph einfuegen */
v1 = g.addV('Auto').property('Modell','Leaf');
v2 = g.addV('Person').property('Name','Robin');
g.V(v1).addE('besitzt').to(v2).property('seit', '2019.08.08');

/* Daten abfragen */
g.V().hasLabel('Person').has('Name','Robin').outE('besitzt').outV().values('Modell')

/* Daten loeschen */
g.V().hasLabel("Person").has('Name', 'Max').drop();
\end{lstlisting}

\subsection{Cypher}
Die Abfragesprache Cypher, wurde mit dem Ziel entworfen, eine einfache deklarative Art der Interaktion mit Graph-Daten zu ermöglichen \cite{gdbms}. Eine wichtige Sprachkomponente stellt dabei der Pattern-Matching-Stil dar \cite{gdbms}. Von den beiden für die Arbeit relevanten Datenbanksystemen Neo4j und Db2 Graph unterstützt allerdings nur Neo4j die Sprache -- zumindest direkt. 

In \autoref{src:cypher_example} werden hierbei einige Beispiel-Cypher-Queries aufgeführt. Diese sind mit den Gremlin-Queries in \autoref{src:gremlin_example} bezüglich ihrer Funktion vergleichbar. 

\begin{lstlisting}[caption={Beispiel Cypher-Queries},language=SQL,label=src:cypher_example]
/* Daten in den Graph einfuegen */
CREATE (:Person{name: "Robin"})-[:besitzt{seit: "2019.08.08"}]->(:Auto{modell: "Leaf"});

/* Daten abfragen */
MATCH (:Person{name: "Robin"})-[:besitzt]->(n2) RETURN n2.modell;

/* Daten loeschen */
MATCH (n:Person{name: "Robin"}) DETACH DELETE n;
\end{lstlisting}

Neben Neo4j das im Kontext der Arbeit eine wichtige Rolle spielt, unterstützen auch die folgenden Datenbanksysteme Cypher als Abfragesprache:
\begin{itemize}
    \item RedisGraph \cite{redisgraph_2021},
    \item SAP HANA Graph \cite{opencypher_2021} und
    \item Agens Graph \cite{opencypher_2021}.
\end{itemize}

\subsection{Zusammenfassung}
Bei den im Rahmen der Arbeit relevanten Abfragesprachen handelt es sich um SQL, Gremlin und Cypher. 

SQL stellt dabei eine Abfragesprache für relationale Datenbanksysteme dar, während Gremlin und Cypher für den Einsatz bei Graphdatenbanksystemen konzipiert wurden. 

Bei SQL und Cypher handelt es sich um deklarative Abfragesprachen. Gremlin hingegen unterstützt neben einem deklarativen auch einen imperativen Abfragestil. 