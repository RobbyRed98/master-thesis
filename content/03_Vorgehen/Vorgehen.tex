\chapter{Vorgehen}
\label{vorgehen}
Im Rahmen dieses Kapitels wird auf das Vorgehen beziehungsweise die Umsetzung der Performance-Analyse der Db2 Graph und Neo4j Datenbanksysteme eingegangen. 

Zuerst wird dabei darauf erörtert, warum sich die Ergebnisse aus \cite{sigmod_tian} im Rahmen dieser Arbeit nicht reproduzieren lassen. Anschließend wird ein neues Ziel für diese Arbeit formuliert, da sich das eigentliche Ziel der Arbeit nicht mehr erfüllen lässt. 

In den darauf folgenden Abschnitten werden einzelne Aspekte der Messungen genauer beschrieben, die für das Verständnis der in \autoref{analyse:messreihen} beschrieben Messreihen notwendig sind.

\section{Reproduktion}
\label{analyse:reproduktion}
Wie in der \nameref{einleitung} beschrieben, war es das eigentliche Ziel dieser Arbeit, die in \cite{sigmod_tian} erzielten Werte für Db2 Graph zu reproduzieren und mit neuen Werten für Neo4j zu vergleichen. 

Allerdings stellte sich nach einigen Recherchen hierbei heraus, dass sich die Werte aus \cite{sigmod_tian} im Rahmen dieser Arbeit nicht reproduzieren lassen. 

Eine Reproduktion der Ergebnisse der Performance-Analyse aus \cite{sigmod_tian} setzt hierbei die folgenden sieben Punkte voraus:
\begin{itemize}
    \item dieselbe Version von Db2 Graph,
    \item dieselbe Db2- und Db2-Graph-Konfiguration,
    \item derselbe Benchmark,
    \item derselbe Db2-Graph-Benchmark-Adapter,
    \item dieselbe Benchmark-Konfiguration,
    \item derselbe Datensatz und 
    \item dieselbe Umgebung (OS, Ressourcen, etc.) wie in \cite{sigmod_tian}.
\end{itemize}
Von diesen sieben Voraussetzungen können im Rahmen der Arbeit jedoch lediglich zwei teilweise erfüllt werden. Bei diesen zwei Voraussetzungen, die sich teilweise erfüllen lassen, handelt es sich um den Benchmark und die Ressourcen. Die Ressourcen werden hierbei in \cite{sigmod_tian} recht genau beschrieben. Darüber hinaus wird in \cite{sigmod_tian} auch aufgeführt, dass Linkbench als Benchmark für die Messungen herangezogen wird. Allerdings wird keine Auskunft darüber gegeben, welche Version beziehungsweise Implementierung von Linkbench für die Messungen genutzt wird.

Bezüglich der anderen Punkte werden im Rahmen von \cite{sigmod_tian} keinerlei Informationen bereitgestellt. So ist beispielsweise nicht bekannt, mit welcher Version von Db2 Graph die Messungen durchgeführt wurden. Es ist allerdings wahrscheinlich, dass es sich weder um die Version Beta 3 noch V11.5.6.0 handelt, die über den Zeitraum der Arbeit verfügbar waren. Schließlich wurden beide Versionen mehrere Monate nach der Arbeit veröffentlicht. 

Über die Benchmark-Konfiguration und die Implementierung des für den Benchmark benötigten Db2-Graph-Benchmark-Adapter ist ebenfalls nichts bekannt. Genau so werden in \cite{sigmod_tian} keine Informationen bezüglich der  Konfiguration der Db2-Instanz und Db2 Graph angegeben, obwohl diese für die Reproduktion der Ergebnisse eine wichtige Rolle spielen. 

Bezüglich des Datensatzes werden hingegen ein paar Eigenschaften in \cite{sigmod_tian} aufgeführt. Wie und ob der dort verwendet Datensatz durch den Daten-Generator des Benchmarks erzeugt werden kann, ist jedoch weiterhin unklar.

\section{Neues Ziel}
Da die Reproduktion der Messungen aus \cite{sigmod_tian} nicht möglich ist, muss ein neues Ziel für den Rahmen dieser Arbeit definiert werden. 

So setzt sich die Arbeit nun zum Ziel, weiterhin eine Performance-Analyse für Db2 Graph durchzuführen und die dabei erzielten Ergebnisse mit denen von Neo4j zu vergleichen. So orientiert sich die Arbeit weiterhin, bei den folgenden Aspekten an \cite{sigmod_tian}:
\begin{itemize}
    \item Linkbench als Benchmark zur Messungen der Performance,
    \item Db2 Graph als gebenchmarktes Datenbanksystem,
    \item Arbeit mit den Datensatz Größen 10 Millionen und 100 Millionen Knoten (Linkbench-10 und Linkbench-100) und
    \item Arbeit mit den Metriken Latenz und Durchsatz.  
\end{itemize}
Allerdings beschreitet die Performance-Analyse dieser Arbeit nun neue Wege, da Sie ihren Schwerpunkt nun auf die Messung der Performance der Datenbanksysteme bei real- und konstant-verteilten Datensätzen legt. 

Eine weitere neue Aufgabe, die aufgrund des neuen Ziels Teil der Arbeit wird, stellt dabei die Suche nach bereits existierenden Linkbench-Adapter-Implementierungen für Db2 Graph und Neo4j dar. 

Sollten jedoch keine der für die Messungen benötigten Adapter-Implementierung verfügbar sein, müssen diese auch im Rahmen der Arbeit selbst implementiert werden. 

Abschließend gilt es darauf hinzuweisen, dass auch wenn beim neuen Ziel einige Elemente aus \cite{sigmod_tian} übernommen werden, die im Rahmen dieser Arbeit und die in \cite{sigmod_tian} erzielten Ergebnisse nicht vergleichbar sind.

\section{Linkbench}
Auch wenn sich das Ziel der Arbeit geändert hat, so wird Linkbench weiterhin als Benchmark zur Performance-Analyse herangezogen. Daher wird eine Implementierung für Linkbench benötigt sowie Adapter für Db2 Graph Beta 3, Db2 Graph V11.5.6.0 und Neo4j. Die Wahl der Linkbench und Adapter Implementierung sowie ihre Umsetzungen werden allerdings in \autoref{implementierung} beschrieben. 

In diesem Abschnitt als Teil des Kapitels \nameref{vorgehen}wird auf die Begriffe: 
\begin{itemize}
    \item Linkbench-Threads,
    \item Request-Anzahl und
    \item Operationsmix.
\end{itemize}
Hierbei handelt es sich um wichtige Konfigurationen von Linkbench. Diese spielen während der \textit{Request}-Phase des Benchmarks eine Rolle. Darüber hinaus stellen sie auch relevante Parameter für die im Rahmen der Performance-Analyse durchgeführten Messungen dar. 

\subsection{Linkbench-Threads}
\label{analyse:linkbench:threads}
Hierbei handelt es sich um eine Einstellung des Benchmarks, die in der Adapter-Konfiguration von Linkbench erfolgt. Innerhalb der Konfiguration wird sie auch durch den Key \texttt{requesters} gesetzt. 

Im Rahmen der Arbeit wird die Konfiguration allerdings als Linkbench-Threads bezeichnet, da durch diese spezifiziert wird, wie viele Threads während des Benchmarks gleichzeitig Requests beziehungsweise Graph-Queries an ein Datenbanksystem senden.

Es handelt sich hierbei somit um eine Einstellung, die einen erheblichen Einfluss auf die bei der Performance-Analyse erzielten Ergebnisse hat. Schließlich hat die Anzahl der Threads, die gleichzeitig während den Messungen Anfragen an ein Datenbanksystem stellen, einen großen Einfluss darauf, welche Last während des Benchmarks auf ein Datenbanksystem ausgeübt wird. 

Bei den Linkbench-Threads handelt es sich darüber hinaus um einen Parameter, der über alle Messreihen der Arbeit hinweg konstant den Wert 50 aufweist. So kann sichergestellt werden, dass während den Messungen alle Datenbanksysteme zumindest in dieser Hinsicht derselben Last ausgesetzt sind. 

\subsection{Request-Anzahl}
\label{Request-Anzahl}
Bei der Request-Anzahl handelt es sich ebenfalls wie bei den Linkbench-Threads um eine Einstellung, die in der Adapter-Konfiguration von Linkbench vorgenommen wird. Um sie festzulegen, muss ein Wert für \texttt{requests} angegeben werden. 

Zum besseren Verständnis wird die Konfiguration allerdings im weiteren Verlauf als Request-Anzahl bezeichnet. Die Request-Anzahl regelt dabei, wie viele Graph-Queries von jeweils einem der Linkbench-Threads an das zu benchmarkende Datenbanksystem gesendet werden, bevor eine \textit{Request}-Phase endet und eine Statistik aufgestellt wird. 

Der Faktor wirkt sich dadurch signifikant darauf aus, wie lange eine \textit{Request-Phase} andauert, bevor Linkbench die Ergebnisse für das untersuchte Datenbanksystem in Form einer Statistik bereitstellt. Zugleich bietet er einen Anhaltspunkt für die Qualität der durchgeführten Messungen. Denn um so mehr Graph-Queries Teil des Benchmarks sind, desto ein detailliertes Bild erhält Linkbench von der Performance der Datenbanksysteme. Darüber hinaus Fall bei einer höheren Anzahl an Messungen, auch Anomalien weniger ins Gewicht, welche die Messergebnisse gegebenenfalls verzerren könnten.

Im Rahmen der in der Arbeit durchgeführten Messreihen musste hierbei ein Kompromiss zwischen der Qualität und der Dauer einer \textit{Request}-Phase von Linkbench gefunden werden. 

So wird bei allen Messungen, die auf Basis eines konstant-verteilten Datensatzes durchgeführt werden, die Request-Anzahl von 500.000 eingesetzt. Diese Anzahl wurde hierbei gewählt, da sie ein detailliertes Bild von der Performance eines Datenbanksystems zeichnet und sich die \textit{Request}-Phase für alle drei Datenbanksysteme unterhalb eines Zeitraums von drei Stunden befindet -- sollte es zu keinen Anomalien kommen. 

Für die Messungen, die auf Basis eines real-verteilten Datensatzes erfolgen, wird lediglich eine Request-Anzahl von 50.000 herangezogen. Dies hängt damit zusammen, dass hierbei deutlich umfangreichere Ergebnismengen möglich sind. So kann es je nach Variation der \texttt{getLinkList}-Operation, zu einer Ergebnismenge von bis 100.000 Links (Kanten) kommen. Eine solch große Ergebnismenge hat hierbei einen signifikanten Einfluss auf die Performance der Datenbanksysteme während des Benchmarks. Dies kann zur Folge haben, dass die Performance eines Datenbanksystems nachweisbar einbricht, besonders im Vergleich zu Messungen auf Basis eines konstant zehn verteilten Datensatzes. Um weiterhin eine Performance von unter drei Stunden für eine Request-Phase gewährleisten zu können, wurde die Request-Anzahl daher reduziert. Auch wenn dadurch nicht dieselbe Qualität der Ergebnisse erreicht werden kann, wie bei den Messungen mit einem konstant verteilten Datensatz.

Die unterschiedliche Request-Anzahl bei verschiedenen stellt hierbei einen wichtigen Punkt dar, der später beim Vergleich Ergebnisse miteinander beachtet werden muss. Schließlich fällt beispielsweise bei Messungen mit einer geringeren Request-Anzahl die Anfangsphase, in der die Datenbanksysteme oft erst warmlaufen, stärker ins Gewicht als bei Messungen mit einer höheren Request-Anzahl.

\subsection{Operationsmix}
\label{analyse:linkbench:operationsmix}
Beim Operationsmix handelt es sich um Einstellungen, die in der Workload-Konfiguration von Linkbench vorgenommen werden. Sie bestimmen dabei, welchen Anteil eine Operation an der Gesamtzahl von Requests während der \textit{Request}-Phase hat. Wird beispielsweise in der Workload-Konfiguration der Parameter \texttt{getnode} auf 100 gesetzt, so handelt es sich bei allen Operationen, die während einer \textit{Request}-Phase durchgeführt werden, um \texttt{getNode}-Operationen.

Im Rahmen der Arbeit wird die Zusammensetzung des Operationsmix dabei immer zwischen allen Messungen eines Datenbanksystems in einer bestimmten Konfiguration variiert. Dabei macht eine Operation beziehungsweise Operationsart immer 100 \% aller im Rahmen einer Messung durchgeführten Requests aus. So wird jede Operationsart wie \texttt{getNode}, \texttt{getLink}, etc. im Rahmen einer separaten \textit{Request}-Phase von Linkbench gemessen und analysiert.

Die Operationen werden dabei getrennt voneinander gemessen, um die während der \textit{Request}-Phase erzielte Ressourcenauslastung immer einer bestimmten Operationsart zuordnen zu können. Wie die Ressourcenauslastung gemessen wird, wird in \autoref{analyse:metriken} genauer beschrieben.

\section{Operationen}
\label{analyse:operationen}
In diesem Abschnitt wird auf die verschiedenen Operationen eingegangen deren Performance im Rahmen der Analyse dieser Arbeit eine Rolle spielen. 

Bei den im Rahmen der Performance-Analyse gebenchmarkten Operationen handelt es sich um dieselben Operationen, die auch in \cite{sigmod_tian} untersucht werden. Dabei gilt es allerdings zu beachten, dass obwohl die Namen der Operationen gleich lauten, nicht bekannt ist, ob die Queries, die von ihnen an die Datenbanksysteme gesendet werden, identisch sind. Daher sollten aus den gleichen Namen keine falschen Schlüsse gezogen werden. 

Des Weiteren ist es auch nicht möglich, die untersuchten Operationen im Vergleich zu \cite{sigmod_tian} auszuweiten. Schließlich handelt es sich lediglich bei diesen um lesende Operationen, die dadurch auch von Db2 Graph als eine Art Read-Only-Datenbanksystem unterstützt werden. Alle weiteren Operationen, die von Linkbench unterstützt werden (\autoref{linkbench:operationen}), sind hingegen von schreibender Natur. 

Bei den im Rahmen der Performance-Analyse gebenchmarkten Operationen handelt es sich hierbei um:
\begin{itemize}
    \item \texttt{getNode},
    \item \texttt{getLink},
    \item \texttt{countLink} und
    \item \texttt{getLinkList}. 
\end{itemize}

Wobei es allerdings anzumerken gilt, dass bei den Messungen mit Bezug zu real-verteilten Datensätzen vier verschiedene Varianten von \texttt{getLinkList} gemessen werden. Bei diesen Varianten variiert jeweils die maximale Größe der sogenannten Ergebnismenge. Dies führt dazu, dass in der Praxis eigentlich die folgenden sieben Operation gebenchmarkt werden:
\begin{itemize}
    \item \texttt{getNode},
    \item \texttt{getLink},
    \item \texttt{countLink},
    \item \texttt{getLinkList(100)},
    \item \texttt{getLinkList(1.000)},
    \item \texttt{getLinkList(10.000)} und
    \item \texttt{getLinkList(100.000)}.
\end{itemize}
Die eingeklammerten Zahlen bei \texttt{getLinkList} repräsentieren hierbei die obere Grenze der Ergebnismenge der Operation. Die Rolle und der Einsatz der Begrenzung der Ergebnismenge wird dabei im nächsten Abschnitt \autoref{analyse:ergebnismenge} im Detail erläutert. 

\section{Begrenzung der Ergebnismenge}
\label{analyse:ergebnismenge}
In diesem Abschnitt wird auf die Rolle der Begrenzung der Ergebnismenge bei der \texttt{getLinkList}-Operation genauer eingegangen. Diese spielt hierbei im Kontext der Performance-Analyse eine wichtige Rolle, da sie bei einigen Messungen eine wichtige Stellung als Messparameter einnimmt, welches es bei der Auswertung der Ergebnisse zu beachten gilt. 

Bei der Begrenzung der Ergebnismenge handelt es sich um eine Workload-Konfiguration, die lediglich die \texttt{getLinkList}-Operation betrifft. Durch sie wird eine obere Grenze für die Anzahl an Elementen einer Ergebnismenge in der Query der \texttt{getLinkList}-Operation gesetzt. Wird die Grenze beispielsweise auf 10 gesetzt, so übermitteln abgefragten die Datenbanksysteme als Antwort auf die \texttt{getLinkList}-Queries maximal eine Menge mit zehn Links (Kanten) an den Benchmark. 

Die Begrenzung der Ergebnismenge wurde dabei im Rahmen dieser Arbeit in den Benchmark eingeführt, um dadurch die Performance der verschiedenen Datenbanksysteme bei größeren oder kleineren Ergebnismengen untersuchen zu können. 

Bei den im Rahmen dieser Arbeit durchgeführten Messungen nimmt die Grenze der Ergebnismenge die folgenden Werte an:
\begin{itemize}
    \item 100,
    \item 1.000,
    \item 10.000 und
    \item 100.000. 
\end{itemize}
Diese wurden dabei ausgewählt, da der Faktor-zehn-Unterschied zwischen den Werten für eine Differenz sorgt, von der angenommen wurde, dass sie ausreicht, um unterschiedliche Messergebnisse für die \texttt{getLinkList}-Operation zu erzielen. 

100.000 wurde darüber hinaus als größte Menge gewählt, da eine Grenze von 1.000.000 den Messzeitraum der \texttt{Request}-Phase von 3 Stunden bei einigen Datenbanksystemen um ein Vielfaches überschritten hätte.

Wie bereits zuvor erwähnt, spielt die Variation der Begrenzung der Ergebnismenge lediglich bei den Messungen mit Bezug real-verteilten Datensätzen eine Rolle. Dies hängt damit zusammen, dass bei konstant-verteilten Datensätzen im Rahmen dieser Arbeit immer eine sogenannte 10er-Verteilung vorherrscht. Dies bedeutet, dass die Ergebnismenge für die \texttt{getLinkList}-Operation maximal 10 Elemente beinhalten kann, egal ob jetzt die obere Grenze für die Größe der Ergebnismenge 100 oder 100.000 beträgt. Somit erübrigt sich die Variation der Begrenzung der Ergebnismenge bei diesen Messungen. 

\section{Queries}
\label{analyse:queries}
Dieser Abschnitt setzt sich mit den Gremlin- und Cypher-Queries auseinander die für die Umsetzung der Messungen herangezogen wurden. Darüber hinaus wird auch auf den Unterschied zwischen den regulären Queries und ID-Queries eingegangen. Der Schwerpunkt dieses Abschnitts liegt dabei allerdings auf den sogenannten regulären Gremlin- und Cypher-Queries, da die ID-Queries im Rahmen dieser Arbeit lediglich eine Nebenrolle spielen. 

Es gilt dabei darauf hinzuweisen, dass die in diesem Abschnitt dargestellten Queries in gewissem Ausmaß der Implementierung der Adapter in \autoref{implementierung} vorgreift. Allerdings nehmen die Queries eine signifikante Rolle bei den Messungen ein. Schließlich handelt es sich bei ihnen um die Anfragen, die bei der Durchführung einer Operation während des Benchmarks zur Interaktion mit den Datenbanksystemen herangezogen wird. Daher werden sie in diesem Abschnitt des Kapitels \autoref{vorgehen} näher beschrieben.

\subsection{Reguläre Queries}
Bei den regulären Queries handelt es sich um die Queries die bei allen bis auf einer Messreihe für die Messungen herangezogen werden. Die entsprechenden Queries die in der Abfragesprache Gremlin formuliert wurden, werden dabei in \autoref{src:gremlin_queries} aufgeführt. Die entsprechenden Queries die in der Abfragesprache Cypher umgesetzt wurden, werden hingegen in \autoref{src:cypher_queries} abgebildet. Im Zuge dessen gilt es darauf hinzuweisen, dass in \autoref{src:gremlin_queries} Platzhalter innerhalb der Queries von \texttt{<} und \texttt{>} umschlossen werden. Bei \autoref{src:cypher_queries} werden sie hingegen mittels eines anführenden \texttt{\$} gekennzeichnet.

Die Gremlin- und Cypher-Queries, welche im \autoref{src:cypher_queries} und \autoref{src:gremlin_queries} aufgeführt werden verfügen darüber hinaus über dieselbe Abfragelogik. Damit ist gemeint, dass beide dieselben Daten abfragen, wodurch sie also äquivalent zueinander sind. Die Gremlin-Queries werden dabei zur Abfrage von Db2 Graph Beta 3 und V11.5.6.0 genutzt, während die Cypher-Queries für Neo4j zum Einsatz kommen. 

\begin{lstlisting}[label=src:gremlin_queries,caption={ Gremlin Queries (regulär)},language=Java]
/* getNode */
g.V()
 .hasLabel("NODETABLE")
 .has("ID", <NODE_ID>);

/* getLink */
g.E()
 .hasLabel("LINKTABLE")
 .has("LINK_TYPE", <LINK_TYPE>)
 .has("ID1", <NODE_ID1>)
 .has("ID2", P.within(<NODE_ID2s>))

/* countLinks */
g.V()
 .hasLabel("NODETABLE")
 .has("ID", <NODE_ID>)
 .outE("LINKTABLE")
 .has("LINK_TYPE", <LINK_TYPE>)
 .count()

/* getLinkList */
g.V()
 .hasLabel("NODETABLE")
 .has("ID", <NODE_ID1>)
 .outE("LINKTABLE")
 .has("LINK_TYPE", <LINK_TYPE>)
 .limit(<LIMIT>);
\end{lstlisting}

\begin{lstlisting}[label=src:cypher_queries,caption={Cypher Queries (regulär)},language=CQL]
/* getNode */
MATCH (n:node{id: $id}) 
RETURN n.id AS ID, n.type AS TYPE, 
    n.version AS VERSION, n.time AS TIME, 
    n.data AS DATA

/* getLink */
MATCH (n1:node{id: $id1})-[l:link{link_type: $link_type}]->(n2:node) 
WHERE n2.id IN $id2s 
RETURN n1.id AS ID1, n2.id AS ID2, 
    l.link_type AS LINK_TYPE, 
    l.visibility AS VISIBILITY, 
    l.data AS DATA, l.time AS TIME, 
    l.version AS VERSION

/* countLinks */
MATCH (:node{id: $id1})-
    [l:link{link_type: $link_type}]->
    (:node) 
RETURN COUNT(l) AS COUNT

/* getLinkList */
MATCH (n1:node{id: $id1})-
    [l:link{link_type: $link_type}]->
    (n2:node) 
RETURN n1.id AS ID1, n2.id AS ID2, 
    l.link_type AS LINK_TYPE, 
    l.visibility AS VISIBILITY, 
    l.data AS DATA, l.time AS TIME, 
    l.version AS VERSION 
LIMIT $limit
\end{lstlisting}

\subsection{ID-Queries}
Bei den ID-Queries handelt es sich um eine Abwandlung der regulären Queries. Ihr Hauptunterschied zu den regulären Queries besteht dabei darin, dass sie sich in der Art der Formulierung der Gremlin-Queries von den regulären Queries unterscheiden. Allerdings gibt es auch einen kleinen Unterschied bezüglich der Abfragelogik von \texttt{getLink}, weshalb bei dieser Operation weder die Gremlin- noch die Cypher-ID-Queries äquivalent zu ihren regulären Pendants sind.

Die ID-Queries spielen dabei im Rahmen der Arbeit lediglich eine untergeordnete Rolle. Sie wurden dabei anfänglich als Messparameter in die Performance-Analyse mit einbezogen, da erste Testmessungen darauf hindeuteten, dass sich ihr Einsatz positiv auf die Performance von Db2 Graph Beta 3 auswirkt. Im Verlauf der Performance-Analyse ließ sich dies allerdings nicht bestätigen. Daher werden die ID-Queries lediglich in einer Messreihe bei der Performance-Analyse herangezogen. 

Beim Vergleich der in \autoref{src:gremlin_id_queries} dargestellten Gremlin-ID-Queries mit den regulären Queries in \autoref{src:gremlin_queries} fällt auf, dass bei den Gremlin-ID-Queries ein parametrisierte Form der Steps \texttt{V()} und \texttt{E()} einsetzt wird. Während bei den regulären Queries stattdessen mit den Filter-Steps wie \texttt{hasLabel} oder \texttt{has} gearbeitet wird.

Des Weiteren unterscheidet sich die Logik der \texttt{getLink}-Queries zwischen ID-Queries und regulären Queries. Bei den ID-Queries wird lediglich auf eine ID für den Endknoten des Links geprüft (\texttt{<NODE\_ID2>} \autoref{src:gremlin_id_queries} und \texttt{\$id2} \autoref{src:cypher_id_queries}). Bei den regulären Queries wird auf eine Menge an IDs für den Endknoten geprüft (\texttt{<NODE\_ID2s>} \autoref{src:gremlin_queries} und \texttt{\$id2s} \autoref{src:cypher_queries}). 

In \autoref{src:cypher_id_queries} wird dabei lediglich die \texttt{getLink}-Query aufgeführt, da es sich bei dieser um die einzige Cypher-Queries handelt die sich zwischen den regulär Cypher-Queries und Cypher-ID-Queries unterscheidet.

\begin{lstlisting}[label=src:gremlin_id_queries,caption={Gremlin ID-Queries},language=Java]
/* getNode */
g.V(["NODETABLE",<NODE_ID>]);

/* getLink */
g.E(["LINKTABLE", <NODE_ID1>, <NODE_ID2>, <LINK_TYPE>])

/* countLinks */
g.V(["NODETABLE",<NODE_ID1>])
    .outE("LINKTABLE")
    .has("LINK_TYPE", <LINK_TYPE>)

/* getLinkList */
g.V(["NODETABLE",<NODE_ID1>])
    .outE("LINKTABLE")
    .has("LINK_TYPE",  <LINK_TYPE>)
    .limit(<LIMIT>)
\end{lstlisting}

\begin{lstlisting}[label=src:cypher_id_queries,caption={Cypher ID-Queries},language=CQL]
/* getLink */
MATCH (n1:node{id: $id1})-[l:link{link_type: $link_type}]->(n2:node{id: $id2}) 
RETURN n1.id AS ID1, n2.id AS ID2, 
    l.link_type AS LINK_TYPE, 
    l.visibility AS VISIBILITY, 
    l.data AS DATA, l.time AS TIME, 
    l.version AS VERSION
\end{lstlisting}

\section{Datensatz}
Die Zusammensetzung des Datensatzes und ihre Verteilung spielen infolge der neuen Zielsetzung der Arbeit eine zentrale Rolle für die Performance-Analyse der Dateisysteme.

In diesem Abschnitt wird daher zuerst auf die beiden Arten der Verteilung der Datensätze eingegangen. Im Anschluss daran wird der Aspekt der Größe eines Datensatzes genauer beleuchtet.

\subsection{Verteilung}
Im Rahmen der Arbeit wird zwischen den beiden folgenden Arten der Verteilung unterschieden:
\begin{itemize}
    \item \textit{konstante Verteilung}\\
    Hierbei besitzt jeder Knoten im Datensatz exakt dieselbe Anzahl an Kanten (Links) wie jeder andere Knoten. Im Rahmen dieser Arbeit wurde bei allen Messungen immer eine konstante 10er-Verteilung gewählt. Dies bedeutet das jeder Knoten genau 10 Kanten aufweist.  
    \item \textit{reale Verteilung}\\
    Dabei ist die Verteilung von Kanten (Links) auf Knoten einer realen Verteilung eines Social-Graphs nachempfunden. Die Anzahl der Kanten, über die ein Knoten verfügt, bewegt sich hierfür bei allen Messungen die in dieser Arbeit durchgeführt werden zwischen 2 und ca. 5.000.000. Im Zuge dessen gibt es einige Knoten mit einer geringen Anzahl an Kanten wie zwei und einzelne Knoten die mehr als 1.000.000 Kanten aufweisen. 
\end{itemize}
Die real-Verteilung wird hierbei im Zuge der Performance-Analyse untersucht, da sie die realitätsnahe Zusammensetzung eines Social-Graphs aufweist. Somit geben die Messergebnisse hierbei einen Einblick darin, wie sich die Datenbanksysteme bei solchen Workloads in der Praxis verhalten könnten.

Die konstante Verteilung wurde entgegen der real-Verteilung nicht aufgrund ihrer Praxis-nähe für die Messungen herangezogen. Sie wurde stattdessen eingeführt, da Db2 Graph Beta 3 bei ersten Messungen Schwierigkeiten mit einer hohen Kantenzahl in real-Verteilten Datensätzen aufwies. Das Benchmarking der \texttt{getLinkList}-Operation verursachte hierbei früher oder später einen beinahe Stillstand des Benchmark-Durchlaufs. 

Dieses Verhalten kann hierbei damit begründet werden, dass sich Db2 Graph Beta 3 damit schwertut, mit großen Ergebnismengen von über 100.000 Kanten je Knoten zu arbeiten. Darüber hinaus kann dieses Problem von Db2 Graph Beta 3 mit real-verteilten Datensätzen auch nicht durch die Begrenzung der Ergebnismenge im Rahmen der Gremlin-Query gelöst werden. Schließlich beherrscht Db2 Graph Beta 3 die Optimierungstechnik \textit{Limit Pushdown} nicht, siehe \autoref{db2graph:optimierung}. 

Aufgrund dessen und da es -- nachvollziehbarer Weise -- nicht zielführend wäre, die maximale Kanten auf Knoten Verteilung eines real-verteilten Datensatzes zu verändern, wurde auch entschieden, konstant-verteilte Datensätze als Teil der Messungen mit aufzunehmen. Schließlich ergibt sich dadurch die Möglichkeit Messergebnisse für Db2 Graph Beta 3 zu erzielen und diese mit V11.5.6.0 und Neo4j zu vergleichen. Allerdings nur unter der Voraussetzung, dass eine kleine konstante Verteilung gewählt wird, wie die konstante 10er-Verteilung. 

Aufgrund des zuvor angesprochenen Problems von Db2 Graph  Beta 3 mit real-verteilten Datensätzen spielt Beta 3 auch bei den Messungen, die auf real-verteilten Datensätzen aufbauen im Kontext dieser Arbeit keine Rolle. Dies bedeutet, dass bei diesen keinerlei Messungen mit Beta 3 durchgeführt werden, lediglich mit Db2 Graph V11.5.6.0 und Neo4j. 

\subsection{Größe}
Die Größe eines Datensatzes stellt neben dessen Verteilung einen großen Unterschied zwischen den im Rahmen der Performance-Analyse durchzuführenden Messungen dar. Dabei wird hier unter der neuen Zielsetzung der Arbeit, ebenfalls mit den aus \cite{sigmod_tian} bekannten Begriffen Linkbench-10M und Linkbench-100M gearbeitet. Die Begriffe beschreiben hier grob eine Datensatzgröße. So stehen die 10M und 100M jeweils dafür, dass ein Datensatz jeweils 10 oder 100 Millionen Knoten beinhaltet. So stellen die Linkbench-10M Datensätze die kleineren Datensätze im Rahmen der Performance-Analyse dar, während die Linkbench-100M Datensätze die Rolle des größeren Datensatzes einnehmen. 

Die Variation der Datensatzgröße über die im Rahmen der Performance-Analyse durchgeführten Messungen stellt hierbei einen interessanten Untersuchungsaspekt dar. Schließlich gilt es bei der Beurteilung von Datenbanksystemen immer zu beachten, wie diese mit kleineren oder größeren Datenmengen umgehen.

Bei den Begriffen Linkbench-10M und Linkbench-100M muss allerdings darauf aufmerksam gemacht werden, dass sich die Datensätze je nach Verteilung weiterhin in ihrer Größe unterscheiden können. So verfügen ein real-verteilter und ein konstant-verteilter Linkbench-10M Datensatz im Rahmen der Arbeit immer über unterschiedliche Größen. Dies hängt damit zusammen, dass sie durch die Verteilung über eine unterschiedliche Anzahl an Kanten verfügt, auch wenn sie gleich viele Knoten besitzen. So weisen die konstant-verteilten Datensätze aus \autoref{tab:kanten_anzahl} fast die doppelte Anzahl an Kanten auf, wie die real-verteilten Datensätze. 

\begin{table}[ht]
    \centering
    \begin{tabular}{l|r|r}
    \hline
    \rowcolor[HTML]{EFEFEF} 
    \multicolumn{1}{c|}{\cellcolor[HTML]{EFEFEF}\textbf{Verteilung}} & \multicolumn{1}{c|}{\cellcolor[HTML]{EFEFEF}\textbf{Linkbench-10M}} & \multicolumn{1}{c}{\cellcolor[HTML]{EFEFEF}\textbf{Linkbench-100M}} \\ \hline
    real & ca. 53.000.000 Kanten & ca. 530.000.000 Kanten \\
    konstant & 100.000.000 Kanten & 1.000.000.000 Kanten \\ \hline
    \end{tabular}
    \caption{Übersicht Linkbench Kantenanzahl}
    \label{tab:kanten_anzahl}
\end{table}

Aufgrund dessen sollte auch später vom Vergleich der Messergebnisse zwischen den unterschiedlich verteilten Datensätzen bei Linkbench-10M und Linkbench-100M abgesehen werden.

\section{Metriken}
\label{analyse:metriken}
Bei den Messungen die im Rahmen der Arbeit durchgeführt werden, werden wie bereits in \cite{sigmod_tian} die Metriken Latenz und Durchsatz für die Auswertung der Ergebnisse herangezogen. 

Der Begriff Latenz bezeichnet dabei den Zeitraum zwischen dem Abschicken einer Anfrage an ein Datenbanksystem und dem Eintreffen der Antwort auf die Anfrage. Alternative kann die Latenz auch als Verarbeitungszeit, also die Zeit, die ein Datenbanksystem für die Verarbeitung einer Anfrage benötigt, betrachtet werden. 

Unter der Metrik Durchsatz wird hingegen während der Performance-Analyse erfasst, wie viele Operationen pro Sekunde durchschnittlich während einer Messung durchgeführt werden konnten. 

Darüber hinaus werden während allen Messungen  Betriebssystemstatistiken erfasst. Die Betriebssystemstatistiken werden dabei durch das Werkzeug NMON erstellt. Dieses zeichnet dabei einmal pro Minute während einer Messung einen Datenpunkt auf. Diese Statistiken geben einem dabei einen Einblick darin, wie groß der Ressourcenverbrauch der Datenbanksysteme während der Messungen ist.

\section{Umgebung}
\label{analyse:umgebung}
Für alle Messungen, die Teil der Performance-Analyse sind, wird im Rahmen der Arbeit derselbe Server herangezogen. Bei diesem handelt es sich um einen Ubuntu-Server 20.04.2 LTS mit folgenden Charakteristika:
\begin{itemize}
    \item 32 CPUs (AMD EPYC 7502P), 
    \item 256 GB Arbeitsspeicher,
    \item 500 GB SSD-Hauptspeicher,
    \item \texttt{ext4} als Dateisystem und 
    \item dem Linux-Kernel \texttt{5.4.0-77-generic}.
\end{itemize}
Des Weiteren gilt es dabei herauszustellen, dass die folgenden Datenbanksysteme während Messungen alle als Docker-Container betrieben werden: 
\begin{itemize}
    \item Db2,
    \item Db2 Graph Beta 3,
    \item Db2 Graph V11.5.6.0 und 
    \item Neo4j. 
\end{itemize}
Jedes dieser Systeme wird in einem eigenen Container betrieben. 

\section{Datenbanksysteme}
\label{analyse:datanbanksysteme}
Im Rahmen der Performance-Analyse werden Messungen mit den folgenden Datenbanksystemen durchgeführt: 
\begin{itemize}
    \item Db2 Graph Beta 3 (+ Db2),
    \item Db2 Graph V11.5.6.0 (+ Db2) und
    \item Neo4j.
\end{itemize}
In den folgenden Unterabschnitten werden dabei die wichtige Anwendungs- be\-zieh\-ungs\-wei\-se System-Konfigurationen beschrieben. 

\subsection{Db2}
Die beiden Db2 Graph Versionen verwenden im Rahmen der Performance-Analyse bei allen Messungen dieselbe Db2-Instanz. Dadurch wird gewährleistet, dass beide auf ein relationales Datenbankmanagementsystem zugreifen, das sich nicht in der Konfiguration zwischen den Db2 Graph Versionen unterscheidet. Außerdem können somit beide Db2 Graph Versionen auf Basis eines identischen Datensatzes operieren. Auf diese Weise kann eine Verzerrung der Messergebnisse vermieden werden und beide Versionen von Db2 Graph arbeiten während des Benchmarking unter denselben Voraussetzungen.

Bei der Db2-Instanz, die von beiden Db2 Graph Versionen genutzt wird, handelt es sich um die Db2 Advanced Edition in Version v11.5.5.1. Per Konfiguration wurden ihr 180 GB an Arbeitsspeicher zugewiesen. Darüber hinaus verfügt sie über die für Linkbench benötigte Datenbank \texttt{linkdb0}. Diese weist im Kontext aller Messungen das in \autoref{src:linkdb0_schema} beschrieben Datenbankschema auf:

\begin{lstlisting}[label=src:linkdb0_schema,caption={Db2-Instanz Datenbankschema für linkdb0},language=SQL]
CREATE TABLE linkdb0.nodetable
(
    id      bigint NOT NULL GENERATED ALWAYS AS IDENTITY (START WITH 1 INCREMENT BY 1),
    type    int       NOT NULL,
    version numeric   NOT NULL,
    time    int       NOT NULL,
    data    clob(48000)  NOT NULL,
    PRIMARY KEY (id)
) ORGANIZE BY ROW COMPRESS YES;

CREATE TABLE linkdb0.linktable
(
    id1        bigint  NOT NULL DEFAULT '0',
    id2        bigint  NOT NULL DEFAULT '0',
    link_type  bigint  NOT NULL DEFAULT '0',
    visibility smallint     NOT NULL DEFAULT '0',
    data       varchar(255) NOT NULL DEFAULT '',
    time       bigint  NOT NULL DEFAULT '0',
    version    bigint       NOT NULL DEFAULT '0',
    PRIMARY KEY (link_type, id1, id2)
) ORGANIZE BY ROW COMPRESS YES;

/* Exisitiert, wird aber nicht genutzt. */
CREATE TABLE linkdb0.counttable
(
    id        bigint NOT NULL DEFAULT '0',
    link_type bigint NOT NULL DEFAULT '0',
    count     int         NOT NULL DEFAULT '0',
    time      bigint NOT NULL DEFAULT '0',
    version   bigint NOT NULL DEFAULT '0',
    PRIMARY KEY (id, link_type)
) ORGANIZE BY ROW COMPRESS YES;
\end{lstlisting}

An dem in \autoref{src:linkdb0_schema} beschriebenen Schema kann dabei abgelesen werden, dass alle Tabellen adaptive Kompression nutzen. Darüber hinaus existiert in \autoref{src:linkdb0_schema} eine Count-Tabelle (\texttt{linkdb0.counttable}) die von den Linkbench-Adaptern für Db2 Graph Beta 3 und V11.5.6.0 nicht genutzt wird. Sie ist lediglich vorhanden, da Linkbench ihre Existenz voraussetzt. 

Neben dem Datenbankschema von \texttt{linkdb0} muss auch noch darauf hingewiesen werden, dass die Datenbank einen Bufferpool (\texttt{IBMDEFAULTBP}) der Größe 163,920 GB nutzt. Der Bufferpool besteht dabei aus exakt 40.000.000 Seiten, mit einer Seitengröße (\texttt{PAGESIZE}) von 4096 Byte.

\subsection{Db2 Graph}
Die Version Beta 3 und V11.5.6.0 von Db2 Graph verfügen im Rahmen der Performance-Analyse eine vergleichbare Konfiguration. Lediglich die in V11.5.6.0 neu eingeführte \texttt{db2graph-server.yaml} und \texttt{db2graph-internal.yaml} sorgen dafür, dass es Konfigurationsunterschiede zwischen beiden Versionen gibt. Auf diese beiden Konfigurationen wird allerdings nicht näher eingegangen. Schließlich wurden die beiden Konfigurationen nicht für die Messungen angepasst. Somit wird im Rahmen der Performance-Analyse die mit Db2 Graph V11.5.6.0 ausgelieferte Standard-Konfiguration von \texttt{db2graph-server.yaml} und \texttt{db2graph-internal.yaml} her\-angezogen.

Die \texttt{gremlin-server.yaml} Konfiguration sowie die Ressourcen-Parameter des Gremlin-Servers von Db2 Graph Beta 3 und V11.5.6.0 wurde beide allerdings gleichermaßen für die Messungen angepasst. Unter diese Anpassungen fallen:
\begin{itemize}
    \item \textit{Gremlin-Memory}\\
    Dieser Ressourcen-Parameter legt fest wie viel Memory der Gremlin-Server des TinkerPop-Stacks im maximal Fall nutzen kann. Der Standard-Wert beträgt hier bei beiden Db2 Graph Versionen normalerweise 4 GB. Um eine möglichst hohe Performance mit beiden Db2 Graph Versionen zu erzielen wurde das Memory-Limit für den Gremlin-Server im Rahmen der Messungen allerdings auf 64 GB angehoben. 
    \item \textit{Thread-Pool-Worker}\\
    Der Wert für Thread-Pool-Worker regelt, wie viele Threads für die Verarbeitung von nicht-blockenden Operationen im Gremlin-Server bereitstehen \cite{tinkerpop_2020}. Der Wert ist dabei normalerweise auf 0 gesetzt, was zur Folge hat, das die Anzahl von verfügbaren Prozessoren herangezogen wird \cite{tinkerpop_2020}. Dies hätte im Rahmen der Messumgebung zur Folge, dass die Thread-Pool-Worker standardmäßig auf 32 gesetzt werden würden. 
    
    Um eine möglichst hohe Performance für die Db2 Graph Versionen bei den Messungen zu gewährleisten, wurde der Wert allerdings auf 128 erhöht. Da es dadurch den Gremlin-Server dahingehen unterstützt, große Lasten besser zu verarbeiten. So ist es dem Gremlin-Server dadurch möglich 50 oder 100 Anfragen, die von den entsprechenden Linkbench-Threads während des Benchmarking gesendet werden, auf einmal zu bearbeiten. 
    \item \textit{Gremlin-Pool}\\
    Der Wert für den Gremlin-Pool spezifiziert, wie viele Threads dem Gremlin-Server für die Verarbeitung von blockenden Operationen zur Verfügung stehen \cite{tinkerpop_2020}. Ähnlich wie Thread-Pool-Worker zieht er in der Standard-Konfiguration die verfügbare Prozessoranzahl (32) heran \cite{tinkerpop_2020}. Um eine höhere Performance zu erreichen, wurde er allerdings ebenfalls auf 128 erhöht. 
    
    Die Erhöhung wurde hierbei allerdings lediglich als eine Art Vorsichtsmaßnahme durchgeführt. Schließlich unterstützt Db2 Graph lediglich lesende Queries. Dies hat zur Folge, dass der Gremlin-Server eigentlich nicht auf blockende Operationen zurückgreifen sollte. Allerdings ist nicht bekannt, ob Db2 Graph in der Praxis wirklich nur auf nicht-blockende Operationen zurückgreift. Daher wurde der Wert des Gremlin-Pools analog zu Thread-Pool-Worker angehoben. 
\end{itemize}

Beide Versionen von Db2 Graph nutzen darüber hinaus die in \autoref{src:db2graph_mapping} dargestellte Graph-Overlay-Konfiguration, zum Mapping der in \autoref{src:linkdb0_schema} beschrieben Tabellen auf eine Graphstruktur. Hierbei wird die \texttt{linkdb0.nodetable} aus \autoref{src:linkdb0_schema} als Vertex-Tabelle in \autoref{src:db2graph_mapping} herangezogen, während die \texttt{linkdb0.linktable} die Rolle einer Edge-Tabelle übernimmt. 
\begin{lstlisting}[label=src:db2graph_mapping,caption={Graph-Overlay-Konfiguration Db2 Graph},language=json]
...
"jdbc_num_conn": 100,
"v_tables": [
    {
        "vid": {
            "prefix": "LINKDB0.NODETABLE",
            "id_cols": [
                "ID"
            ]
        },
        "table_id": "LINKDB0.NODETABLE",
        "table": {
            "schema_name": "LINKDB0",
            "table_name": "NODETABLE"
        },
        "label": {
            "fixed_label": true,
            "label": "NODETABLE"
        }
    }
],
"e_tables": [
    {
        "src_v_cols": [
            "ID1"
        ],
        "dst_v_cols": [
            "ID2"
        ],
        "src_v_tables": [
            "LINKDB0.NODETABLE"
        ],
        "dst_v_tables": [
            "LINKDB0.NODETABLE"
        ],
        "eid": {
            "implicit_id": false,
            "id": {
                "prefix": "LINKDB0.LINKTABLE",
                "id_cols": [
                    "LINK_TYPE",
                    "ID1",
                    "ID2"
                ]
            }
        },
        "table_id": "LINKDB0.LINKTABLE",
        "table": {
            "schema_name": "LINKDB0",
            "table_name": "LINKTABLE"
        },
        "label": {
            "fixed_label": true,
            "label": "LINKTABLE"
        }
    }
]
\end{lstlisting}

\subsection{Neo4j}
Alle Messungen die im Rahmen der Performance-Analyse durchgeführt werden, nutzen dieselbe Neo4j-Instanz. Bei dieser Instanz handelt es sich um die Community-Edition von Neo4j in Version 4.3. An der Konfiguration von Neo4j ändert sich zwischen den verschiedenen Messreihen nichts. Sie weist somit immer dieselbe Konfiguration auf. Bei dieser Konfiguration gilt es die folgenden Parameter hervorzuheben: 
\begin{itemize}
    \item 160 GB Page Cache und
    \item maximaler JVM Heap 64 GB.
\end{itemize}
Der Page Cache kann hierbei als das Neo4j-Pendant zum Bufferpool von Db2 betrachtet werden. Daher wurde mit ca. 164 GB und 160 GB auch eine vergleichbare Konfiguration für beide gewählt. Der JVM Heap in Neo4j und der Gremlin-Memory von Db2 Graph wurden hierbei aufgrund ihrer ähnlichen Funktion ebenfalls als vergleichbar angesehen, weshalb beide auf 64 GB gesetzt wurden.

Bei Neo4j handelt es sich um eine native Graphdatenbank, die auf dem in \autoref{datenmodelle} beschrieben Graphmodell basiert. Dadurch unterstützt Neo4j ein flexibles Datenbankschema, wie in \autoref{datenmodelle:structure}bereits beschrieben. Neo4j unterstützt allerdings trotz des flexiblen Schemas die Definition von Constraints. Ein solcher Constraint und ein Index werden hierbei in den Graphdatenbank eingerichtet, um einen fairen Vergleich von zwischen den Db2graph Versionen (+ Db2) und Neo4j zu ermöglichen. Der in \autoref{src:neo4j_schema} beschriebene Constraint und Index werden hierbei eingeführt, um dafür zu sorgen, dass ähnliche Indexe wie in Db2 auch in Neo4j existieren. Schließlich werden in Db2 beim Anlegen eines Primärschlüssels automatisch Indexe für dessen Bestandteile erzeugt. Zum Ausgleich werden daher die \autoref{src:neo4j_schema} spezifizierten Konstrukte angelegt. Der Constraint in \autoref{src:neo4j_schema} sorgt dabei dafür, dass keine Knoten mit dem Label \texttt{node}, dieselbe ID aufweisen können. Zugleich wird im Zuge dessen auch ein Index auf die ID von \texttt{node}-gelabelten Knoten erstellt.

\begin{lstlisting}[label=src:neo4j_schema,caption={Neo4j Instanz Datenbankschema},language=CQL]
// Constraint aehnlich einem Primaerschluessel fuer Knoten
CREATE CONSTRAINT unique_node_id 
ON (n:node) ASSERT n.id IS UNIQUE;

// Index fuer die Kanten-Property link_type
CREATE INDEX link_type_index 
FOR ()-[l:link]-() 
ON (l.link_type);
\end{lstlisting}

\section{Messreihen}
\label{analyse:messreihen}
Um das neue Ziel der Arbeit zu erfüllen, werden in diesem Kapitel verschiedene Messreihen spezifiziert. Eine Messreihe stellt hierbei eine Art Messszenario für die Datenbanksysteme dar. So können die jeweiligen Ergebnisse eines Datenbanksystems immer direkt mit den Ergebnissen der anderen Datenbanksysteme verglichen werden. Schließlich werden die Ergebnisse im Rahmen einer Messreihe immer unter den gleichen oder vergleichbaren Bedingungen erzielt. 

Bevor damit allerdings auf die Unterschiede beziehungsweise die variierenden Parameter der Messreihe eingegangen werden kann, werden hier nochmals kurz die Parameter und Konfigurationen zusammengefasst, die über alle Messreihen hinweg konstant bleiben:
\begin{itemize}
    \item \textit{Umgebung}\\
    In der in \autoref{analyse:umgebung} beschrieben Umgebung, werden alle Messreihen durchgeführt. 
    \item \textit{Linkbench-Threads}\\
    Linkbench arbeitet in der Request-Phase immer mit 50 Threads die gleichzeitig Anfragen an ein Datenbanksystem schicken.
    \item \textit{Db2-Datenbankschema}\\
    Das Db2-Datenbankschema von \texttt{linkdb0} aus \autoref{src:linkdb0_schema} wird zwischen den Messreihen nicht verändert oder angepasst. 
    \item \textit{Konfiguration Db2 Graph Beta 3 \& V11.5.6.0}\\
    Die Graph-Overlay-Konfiguration in \autoref{src:db2graph_mapping} und die Gremlin-Server-Konfiguration der Db2 Graph Versionen bleiben über alle Messungen hinweg unverändert.
    \item \textit{Konfiguration \& Datenbankschema Neo4j}\\
    Die in Konfiguration von Neo4j und das bei den Messungen eingesetzte Datenbankschema \autoref{src:neo4j_schema} bleiben über die Messreihen hinweg unverändert. 
\end{itemize}

In den folgenden Unterabschnitten wird nun der Aufbau der  verschiedenen Messreihen genauer erläutert. Der jeweilige Name einer Reihe setzt sich dabei aus dem Namen des Benchmarks (Linkbench), der Abkürzung für die gewählte Verteilung (Const oder Real) des Datensatzes und der Größe des Datensatzes (10M oder 100M) zusammen. Falls bei einer Messreihe ID-Queries statt den regulären Queries zum Einsatz kommen, wird abschließend noch ID angehängt.

\subsection{Linkbench-Const-10M}
Im Rahmen dieser Messreihe werden Messungen mit den Datenbanksystemen:
\begin{itemize}
    \item Db2 Graph Beta 3,
    \item Db2 Graph V11.5.6.0 und 
    \item Neo4j durchgeführt. 
\end{itemize}
Die Messungen werden dabei auf Basis der folgenden Eigenschaften beziehungsweise Parameter durchgeführt:
\begin{itemize}
    \item reguläre Queries,
    \item Request-Anzahl (pro Thread) von 500.000, 
    \item konstante Verteilung (Datensatz) und
    \item Linkbench-10M (Datensatzgröße).
\end{itemize}
Wie bei den Parametern beschrieben werden im Rahmen dieser Messreihe die regulären Queries herangezogen. Der für die Messungen erzeugte Datensatz besteht hingegen aus 10 Millionen Knoten und 100 Millionen Kanten. Darüber hinaus weist er eine konstante 10er-Verteilung auf. Das bedeutet jeder Knoten verfügt über exakt 10 ausgehende Kanten. Als Request-Anzahl wird hierbei 500.000 (pro Thread) angewendet.

\subsection{Linkbench-Const-10M-ID}
Teil dieser Messreihe sind Messungen mit den Datenbanksystemen:
\begin{itemize}
    \item Db2 Graph Beta 3,
    \item Db2 Graph V11.5.6.0 und 
    \item Neo4j. 
\end{itemize}

Zugleich finden allen Messungen im Rahmen der Messreihe auf der Basis der folgenden Parameter statt:
\begin{itemize}
    \item Queries-ID,
    \item Request-Anzahl (pro Thread) von 500.000, 
    \item konstante Verteilung (Datensatz) und
    \item Linkbench-10M (Datensatzgröße).
\end{itemize}
Die Messreihe ähnelt in ihrer Beschaffenheit der Messreihe Linkbench-Const-10M. Der Datensatz bei den Messungen weist eine konstante 10er-Verteilung auf und setzt sich aus 10 Millionen Knoten und 100 Millionen Kanten zusammen. Darüber hinaus wurde hier auch eine Request-Anzahl von 500.000 (pro Thread) herangezogen.

Der einzige Unterschied zu der Linkbench-Const-10M Messung stellt hierbei der Einsatz der ID-Queries statt der regulären Queries dar.

\subsection{Linkbench-Const-100M}
Im Kontext dieser Reihe erfolgen Messungen mit den Datenbanksystemen:
\begin{itemize}
    \item Db2 Graph Beta 3,
    \item Db2 Graph V11.5.6.0 und 
    \item Neo4j. 
\end{itemize}

Hierbei werden folgende Parameter bei allen Messungen herangezogen:
\begin{itemize}
    \item reguläre Queries,
    \item Request-Anzahl (pro Thread) von 500.000, 
    \item konstante Verteilung (Datensatz) und
    \item Linkbench-100M (Datensatzgröße).
\end{itemize}
Bei allen Messungen, die Teil dieser Messreihe sind, wird ein konstant (10) verteilter Datensatz eingesetzt, welcher 100 Millionen Knoten und 1 Milliarde Kanten umfasst. Des Weiteren kommen bei diesen Messungen die regulären Queries zum Einsatz. Die Request-Anzahl pro Thread beträgt hier ebenfalls 500.000.

\subsection{Linkbench-Real-10M}
Bei dieser Messreihe werden Messungen mit den Datenbanksystemen:
\begin{itemize}
    \item Db2 Graph V11.5.6.0 und 
    \item Neo4j durchgeführt. 
\end{itemize}

Die Messungen im Rahmen dieser Messreihe werden dabei mit folgenden Parametern durchgeführt:
\begin{itemize}
    \item reguläre Queries,
    \item Request-Anzahl (pro Thread) von 50.000, 
    \item reale Verteilung (Datensatz) und 
    \item Linkbench-10M (Datensatzgröße).
\end{itemize}
Bei den im Kontext der Reihe durchgeführten Messungen werden die regulären Queries eingesetzt. Außerdem wird bevor die Messungen vorgenommen werden können, ein Datensatz mit einer realen Verteilung erzeugt. Der Datensatz setzt sich dabei aus 10 Millionen Knoten und ca. 53 Millionen Kanten zusammen. Als Request-Anzahl pro Thread wurde hier 50.000 gewählt. 

\subsection{Linkbench-Real-100M}
Teil dieser Reihe sind Messungen mit den Datenbanksystemen:
\begin{itemize}
    \item Db2 Graph V11.5.6.0 und 
    \item Neo4j. 
\end{itemize}

Außerdem finden allen Messungen im Kontext der Messreihe auf der Basis der folgenden Parameter statt:
\begin{itemize}
    \item reguläre Queries,
    \item Messumfang von 2.500.000,
    \item reale Verteilung (Datensatz) und
    \item Linkbench-100M (Datensatzgröße).
\end{itemize}
Die Messreihe ähnelt hier im Aufbau der Linkbench-Real-10M Reihe stark. So werden bei dieser ebenfalls reguläre Queries eingesetzt, die Request-Anzahl beträgt 50.000, es werden mehrere Messungen für die \texttt{getLinkList}-Operation durchgeführt und der Datensatz auf dem die Messungen basieren ist auch real verteilt. Allerdings umfasst der hierbei verwendete Datensatz 100 Millionen Knoten und ca. 530 Millionen Kanten, statt den 10 Millionen Knoten und ca. 53 Millionen Kanten bei der Messreihe Linkbench-Real-10M.

\subsection{Übersicht}
In diesen Abschnitt werden alle Messreihen und ihre Parameter nochmals übersichtlich in den beiden Tabellen \autoref{tab:uebersicht_messreihen_datensatz} und \autoref{tab:uebersicht_messreihen_parameter} zusammengefasst. In \autoref{tab:uebersicht_messreihen_datensatz} werden dabei die Datensatz-spezifischen Parameter der Messreihen aufgeführt, wie Größe und Verteilung. Die Parameter Messumfang, Anzahl gebenchmarkter Operationen und sowie Art der Queries werden hingegen in \autoref{tab:uebersicht_messreihen_parameter} aufgelistet.

Der Parameter Messumfang beschreibt hierbei, wie viele Operationen insgesamt während einer \textit{Request}-Phase an von den Linkbench-Threads durchgeführt werden. Also im Rahmen dieser Arbeit immer das 50-fache der Request-Anzahl. Da wie bereits zuvor beschrieben während allen Messungen 50 Linkbench-Threads zum Einsatz kommen. 

\begin{table}[h]
    \centering
    \begin{tabular}{l|r|r|c}
    \hline
    \rowcolor[HTML]{EFEFEF} 
    \multicolumn{1}{c|}{\cellcolor[HTML]{EFEFEF}{\color[HTML]{333333} }} & \multicolumn{3}{c}{\cellcolor[HTML]{EFEFEF}\textbf{Datensatz}} \\ \cline{2-4} 
    \rowcolor[HTML]{EFEFEF} 
    \multicolumn{1}{c|}{\multirow{-2}{*}{\cellcolor[HTML]{EFEFEF}{\color[HTML]{333333} \textbf{Messreihe}}}} & \multicolumn{1}{c|}{\cellcolor[HTML]{EFEFEF}\textbf{Knoten}} & \multicolumn{1}{c|}{\cellcolor[HTML]{EFEFEF}\textbf{Kanten}} & \multicolumn{1}{l}{\cellcolor[HTML]{EFEFEF}\textbf{Verteilung}} \\ \hline
    Linkbench-Const-10M & 10.000.000 & 100.000.000 & konstant \\
    Linkbench-Const-10M-ID & 10.000.000 & 100.000.000 & konstant \\
    Linkbench-Const-100M & 100.000.000 & 1.000.000.000 & konstant \\
    Linkbench-Real-10M & 10.000.000 & 100.000.000 & real \\
    Linkbench-Real-100M & 100.000.000 & 1.000.000.000 & real \\ \hline
    \end{tabular}
    \caption{Übersicht Datensatzparameter der Messreihen}
    \label{tab:uebersicht_messreihen_datensatz}
\end{table}

\begin{table}[h]
    \centering
    \begin{tabular}{l|r|c|c}
    \hline
    \multicolumn{1}{c|}{\cellcolor[HTML]{EFEFEF}{\color[HTML]{333333} }} & \multicolumn{1}{l|}{\cellcolor[HTML]{EFEFEF}} & \cellcolor[HTML]{EFEFEF} & \multicolumn{1}{c}{\cellcolor[HTML]{EFEFEF}} \\
    \multicolumn{1}{c|}{\multirow{-2}{*}{\cellcolor[HTML]{EFEFEF}{\color[HTML]{333333} \textbf{Messreihe}}}} & \multicolumn{1}{l|}{\multirow{-2}{*}{\cellcolor[HTML]{EFEFEF}\textbf{Messumfang}}} & \multirow{-2}{*}{\cellcolor[HTML]{EFEFEF}\textbf{\begin{tabular}[c]{@{}c@{}}Gebenchmarkte \\ Operationen\end{tabular}}} & \multicolumn{1}{c}{\multirow{-2}{*}{\cellcolor[HTML]{EFEFEF}\textbf{Queries}}} \\ \hline
    Linkbench-Const-10M & 25.000.000 & 4 & regulär \\
    Linkbench-Const-10M-ID & 25.000.000 & 4 & ID \\
    Linkbench-Const-100M & 25.000.000 & 4 & regulär \\
    Linkbench-Real-10M & 2.500.000 & 7 & regulär \\
    Linkbench-Real-100M & 2.500.000 & 7 & regulär \\ \hline
    \end{tabular}
    \caption{Übersicht Parameter der Messreihen}
    \label{tab:uebersicht_messreihen_parameter}
\end{table}

\section{Ablauf der Messungen}
Im Rahmen dieses Abschnitts wird darauf eingegangen, wie die Messungen der Datenbanksysteme als Teil der in \autoref{analyse:messreihen} beschriebenen Messreihen durchgeführt werden. 

Die Durchführung der Messreihen läuft hierbei so ab, dass für jeweils eines der zwei bis drei Datenbanksysteme alle der in \autoref{analyse:operationen} beschrieben und benötigten Operationen in einer separaten Messung gebenchmarkt werden. 

Nach dem alle vier bis sieben Operationen, die gebenchmarkt werden müssen, gebenchmarkt wurden, kann mit dem nächsten Datenbanksystem weitergemacht werden. 

Wurden alle Operationen für die Datenbanksysteme, die in einer Messreihe eine Rolle spielen durchgeführt, so kann mit der Durchführung der nächsten Messreihe begonnen werden. 

\todo{Hier ein Tikz Bild einfügen.}