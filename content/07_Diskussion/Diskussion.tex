\chapter{Diskussion}
\label{diskussion}
Im Rahmen der Performance-Analyse und des Vergleichs von Db2 Graph (Beta 3 und V11.5.6.0) als hybridem Graphdatenbanksystem mit Neo4j als nativem Graphdatenbanksystem wurde festgestellt, dass Neo4j eine sechs- bis siebenfach höhere Performance aufweist als Db2 Graph. Des Weiteren wurde ein signifikanter Performance-Unterschied zwischen Db2 Graph Beta 3 und V11.5.6.0 identifiziert, dessen Ursprung im Rahmen der Arbeit nicht geklärt werden konnte. Dabei wies Db2 Graph Beta 3 bei einigen Messungen eine höhere Performance auf als die neuere Version V11.5.6.0, obwohl V11.5.6.0 nachweisbar den performanteren SQL-Code zur Abfrage von Db2 erzeugte. 

Das Ziel der Masterarbeit, eine Performance-Analyse mit Db2 Graph und Neo4j durchzuführen und deren Ergebnisse miteinander zu vergleichen konnte im Rahmen der Arbeit erreicht werden. Jedoch war die dabei angestrebte Reproduktion der Messergebnisse von \citeAY{sigmod_tian} nicht möglich, da hierfür nicht genügend Informationen zur Verfügung stehen. 

Die im Rahmen der Arbeit erzielten Ergebnisse für die Performance von Db2 Graph und Neo4j weichen daher auch erheblich von den von \citeAY{sigmod_tian} beschriebenen Messergebnissen ab. So bewegen sich beispielsweise die Performance-Ergebnisse im Artikel von \citeAY{sigmod_tian} bei den durchschnittlichen Latenzen in einer geringeren Größenordnung als bei den hier erzielten Ergebnissen. Dies lässt sich möglicherweise darauf zurückführen, dass bei den Messungen von \citeAY{sigmod_tian} ein nicht von Linkbench generierter Datensatz oder eine andere Version von Db2 Graph herangezogen wurde. Auf Basis der im Rahmen dieser Arbeit erzielten Messergebnisse lässt sich allerdings festhalten, dass Db2 Graph -- egal ob Beta 3 oder V11.5.6.0 -- bei allen Messungen nicht mit der Performance von Neo4j mithalten kann. Dies ist in der Retrospektive jedoch nicht besonders überraschend. Schließlich handelt es sich bei Db2 Graph als Anwendung lediglich um eine Grapherweiterung für Db2, ein relationales Datenbanksystem, das nicht für den Umgang mit Graphdaten und -Queries optimiert ist. 

Es gilt hier allerdings auch herauszustellen, dass Db2 Graph trotz seiner geringeren Performance im Vergleich zu Neo4j beeindruckendes leistet. Es bietet schließlich die Möglichkeit, die Informationen in einer Db2-Datenbank, auf eine Graphstruktur zu mappen, ohne dass Änderungen am Datenbankschema in Db2 vorgenommen werden müssen. So kann aus den Ergebnissen dieser Arbeit geschlussfolgert werden, dass Db2 Graph in Kombination mit Db2 als hybrides Graphdatenbanksystem nicht mit Neo4j mithalten kann, allerdings als Grapherweiterung für Db2 trotz allem einen Mehrwert bietet. 

Bei der Bewertung und Einordnung der Arbeit selbst gilt es zu beachten, dass die bei der Performance-Analyse erzielten Ergebnisse aufgrund der von Linkbench unterstützten Operationen hauptsächlich \acs{oltp}-Prozesse abbilden. Dadurch wird der Fokus dieser Arbeit auf grundlegende Operationen wie \texttt{getNode} oder \texttt{getLink}, die Knoten und Kanten in einem Graphen abfragen, gelegt. Komplexere \acs{olap}-Prozesse werden allerdings nicht berücksichtigt, da dies den Rahmen der Arbeit überschreiten würde. 

Zudem wird bei den Messungen ein einfaches Datenschema genutzt, bei dem Knoten und Kanten in Db2 in einer Node- und einer Link-Tabelle abgelegt werden, während alle Knoten und Kanten in Neo4j jeweils als \texttt{node} oder \texttt{link} gelabelt werden. Dieses einfache Schema teilt dabei, wie bei relationalen Datenbanksystemen üblich, die Daten in die Entitätstypen Node und Link ein und hält oder labelt sie dementsprechend. Ob es sich hierbei allerdings um das einzige Schema handelt, das für die Performance-Analyse von Neo4j und Db2 Graph geeignet ist, ist hingegen unklar. Schließlich wäre an dieser Stelle auch ein am Graphmodell orientiertes Schema denkbar, bei dem die Nodes und Links auf Basis ihres \texttt{TYPE} und \texttt{LINK\_TYPE} in verschiedene Tabellen eingeordnet werden oder Labels erhalten. Auf das Datenbankschema in Db2 würde sich dies dann wie in \autoref{src:linkdb0_schema_graph_style} beschrieben auswirken. 
\begin{lstlisting}[label=src:linkdb0_schema_graph_style,caption={Alternatives Datenbankschema für Db2},language=SQL]
CREATE TABLE linkdb0.nodetable_<TYPE_0>
(
    id      bigint NOT NULL GENERATED ALWAYS AS IDENTITY (START WITH 1 INCREMENT BY 1),
    version numeric   NOT NULL,
    time    int       NOT NULL,
    data    clob(48000)  NOT NULL,
    PRIMARY KEY (id)
) ORGANIZE BY ROW COMPRESS YES;

CREATE TABLE linkdb0.nodetable_<TYPE_1>(
    ...
) ORGANIZE BY ROW COMPRESS YES;

...

CREATE TABLE linkdb0.linktable_<LINK_TYPE_0>
(
    id1        bigint  NOT NULL DEFAULT '0',
    id2        bigint  NOT NULL DEFAULT '0',
    visibility smallint     NOT NULL DEFAULT '0',
    data       varchar(255) NOT NULL DEFAULT '',
    time       bigint  NOT NULL DEFAULT '0',
    version    bigint       NOT NULL DEFAULT '0',
    PRIMARY KEY (link_type, id1, id2)
) ORGANIZE BY ROW COMPRESS YES;

CREATE TABLE linkdb0.linktable_<LINK_TYPE_1>(
    ...
) ORGANIZE BY ROW COMPRESS YES;

...
\end{lstlisting}
Die Beantwortung der Frage -- \textit{Ob das eingesetzte Datenbankschema bei Db2 Graph (und Db2) sowie bei Neo4j, das optimale Datenbankschema für die Struktur der Daten bei Linkbench ist?} -- hätte allerdings den Umfang dieser Arbeit überschritten. Dasselbe gilt dabei auch für das Durchführen der Performance-Messung mit verschiedenen Datenbankschemen. Jedoch offenbaren die beiden angesprochenen Aspekte Bereiche mit Forschungspotenzial.

Des Weiteren hat die Arbeit Erkenntnisse darüber geliefert, dass sich die Ergebnisse von \citeAY{sigmod_tian} nicht reproduzieren lassen. Db2 Graph weist dabei im Rahmen dieser Arbeit weitaus höhere Latenzen auf als im Artikel von \citeAY{sigmod_tian}. Neben Db2 Graph werden von \citeAY{sigmod_tian} allerdings auch Ergebnisse für JanusGraph aufgeführt. So wäre eine Empfehlung für die weiterführende Forschung, eine Performance-Analyse nach dem Muster dieser Arbeit mit JanusGraph durchzuführen und abschließend einen Vergleich mit den hier erzielten Ergebnissen für Db2 Graph und Neo4j zu ziehen.