\chapter{Fazit}

Neo4j weist mit Abstand die beste Performance aller untersuchten Datenbanksysteme auf, gefolgt von Db2 Graph Beta 3 -- allerdings ausschließlich bei Messreihen mit real verteilen Datensätzen -- und mit größerem Abstand Db2 Graph V11.5.6.0.

Die Performance von Db2 Graph V11.5.6.0 und Neo4j bricht mit steigenden Ergebnismengen bei beiden erheblich ein. Wobei gemessen an der eigenen Leistung Neo4j etwas stärker einbricht als Db2 Graph V11.5.6.0.

Die Datensatzgröße hat bei allen Datenbanksystemen eine relativ geringen Einfluss auf die Performance. So betragen die Abweichungen gemessen an der eigenen Leistung  gerade einmal 0,09 - 5,23 \% zwischen Linkbench-10M und Linkbench-100M Datensätzen.  

Bei der Ressourcenauslastung gilt es anzumerken das Neo4j während der Messungen die höchste CPU-Auslastung gefolgt von Db2 Graph Beta 3 und mit größerem Abstand V11.5.6.0 aufweist. Bei der IO-Auslastung weist Db2 V11.5.6.0 die höchsten Werte auf, gefolgt von Db2 Graph und Neo4j. Neo4j hat dabei eine so geringe IO-Auslastung, dass die Vermutung naheliegt, dass dieses ein weitaus aggressiveres Caching betreibt als Db2, auf das die Db2 Graph Versionen zurückgreifen. 

Db2 Graph V11.5.6.0 erzeugt auf Basis der während der Performance-Analyse gestellten Gremlin-Queries performanteren SQL-Code als Beta 3. Dies scheint somit nicht der Grund dafür zu sein, dass Db2 Graph Beta 3 in allen Messungen, die mit ihm durchgeführt werden, eine höhere Performance als Db2 Graph aufweist. 


\todo{real verteilten -> real verteilten}
\todo{konstant verteilten -> konstant verteilten}
\listoftodos