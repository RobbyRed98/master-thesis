\chapter{Fazit}
\label{fazit}
Anhand der Durchführung von Linkbench Messungen setzte sich diese Masterarbeit im Wesentlichen mit dem Thema der Performance-Analyse von Db2 Graph und Neo4j auseinander. Die zu Beginn der Arbeit angestrebte Reproduktion der Werte von \citeAY{sigmod_tian} war allerdings aufgrund fehlender Informationen nicht möglich. Aus diesem Grund wurden schließlich neue umfangreichere Messungen mit dem Benchmark Linkbench und den Graphdatenbanksystemen Db2 Graph Beta 3, Db2 Graph V11.5.6.0 und Neo4j durchgeführt. 

Bei allen im Rahmen der Arbeit getätigten Messungen weist dabei Neo4j mit großem Abstand die höchste Performance auf. Damit liegt es direkt vor Db2 Graph Beta 3 auf Rang zwei, bei den Messungen konstant verteilten Datensätzen, welches wiederum vor Db2 Graph V11.5.6.0 eingeordnet werden kann. Die höhere Performance von Neo4j äußert sich bei den Messungen in Form der niedrigsten durchschnittlichen Latenz- und höchsten Durchsatzwerten. Gemessen an V11.5.6.0, der neusten Version von Db2 Graph, weist Neo4j eine ungefähr sechs- bis siebenfach höhere Performance auf. Für Db2 Graph Beta 3 werden keine Messungen mit real verteilten Datensätzen durchgeführt, da Beta 3 aufgrund der fehlenden Unterstützung für die \textit{Limit Pushown} Optimierung, den vorgesehen Zeitraum bei den Linkbench Messungen deutlich überschreitet.

Im Zuge der Analyse von Messergebnissen, bei der die obere Grenze der Anzahl an Elementen in einer Ergebnismenge zwischen 100 und 100.000 variiert wird, kann identifiziert werden, dass beide  untersuchten Datenbanksysteme -- Db2 Graph V11.5.6.0 und Neo4j -- erhebliche Performance-Einbrüche bei einer steigenden Ergebnismenge erleiden. Wobei Neo4j im Verhältnis zu seiner eigenen Leistung in den Messungen etwas stärker einbricht als Db2 Graph V11.5.6.0. 

Die bei den Messungen variierte Datensatzgröße hat bei den Datenbanksystemen Db2 Graph Beta 3, Db2 Graph V11.5.6.0 und Neo4j lediglich einen geringen Einfluss auf deren Performance. So bewegt sich der Performance-Unterschied zwischen einem Linkbench-10M und Linkbench-100M im Bereich von 0,09 \% bis 5,23 \%.

Bei dem Vergleich der Performance zwischen Db2 Graph Beta 3 und V11.5.6.0 fällt auf, dass Beta 3 als ältere Version im Schnitt über die 2,25-fache Performance von V11.5.6.0 verfügt -- bei Messungen mit konstant verteilten Datensätzen. Dadurch ergibt sich ein erheblicher Performance-Unterschied zwischen beiden Versionen, dessen Ursprung im Kontext der Arbeit nicht identifiziert werden kann. So ist der von V11.5.6.0 erzeugte und an Db2 gesendete SQL-Code sogar performanter als der von Beta 3, obwohl die von Linkbench erzielten Messergebnisse das Gegenteil präsentieren. Daraus kann geschlussfolgert werden, dass der Grund für den Performance-Unterschied hier bei Db2 Graph als Anwendung liegt und nicht bei Db2 als zugrundeliegendes Datenbanksystem.

Das Ziel der Arbeit, die Ergebnisse von \citeAY{sigmod_tian} für Db2 Graph zu reproduzieren und mit neuen Ergebnissen für Neo4j zu vergleichen kann im Rahmen dieser Arbeit nicht vollständig erfüllt werden. Es wird allerdings trotzdem eine umfangreiche Performance-Analyse durchgeführt, deren Ergebnisse sich für eine Einordnung und den Vergleich der Datenbanksysteme eignen. Dabei weist Neo4j die bessere Performance als Db2 Graph auf. Aber auch wenn Db2 Graph nicht mit der Performance von Neo4j mithalten kann, bedeutet dies nicht, dass sich der Einsatz von Db2 Graph nicht lohnt. So eignet es sich auch mit einer geringeren Performance als Neo4j dazu, seine Aufgabe als Grapherweiterung für Db2 wahrzunehmen.