%----------------------------------------------------------
% Styles
%----------------------------------------------------------

\newcommand{\code}[1]{{\ttfamily #1}}
\newcommand{\name}[1]{{,,#1``}}
\newcommand{\emphasize}[1]{{\itshape #1}}

%----------------------------------------------------------
% Font
%----------------------------------------------------------
\usepackage{lmodern}

% Change font to sans serif
%\renewcommand{\familydefault}{\sfdefault}

%----------------------------------------------------------
% Source code (listings)
%----------------------------------------------------------

% Code style colors
\definecolor{lstbgr}{RGB}{245,245,245}
\definecolor{lstkeyword}{RGB}{0,0,255}
\definecolor{lststring}{RGB}{163,21,21}
\definecolor{lstcomment}{RGB}{0,128,0}

% Default format
\lstset{
	language=Python,%						the language of the code
	backgroundcolor=\color{lstbgr},%		choose the background color
	basicstyle=\ttfamily\footnotesize,%		the size of the fonts that are used for the code
	keywordstyle=\color{lstkeyword},%		keyword style
	stringstyle=\color{lststring},%			the style that is used for strings
	commentstyle=\color{lstcomment},%		comment style
	numberstyle=\ttfamily\footnotesize,%	the style that is used for the line-numbers
	tabsize=2,%								sets default tabsize to 2 spaces
	keepspaces=true,%						keeps spaces in text
	showstringspaces=false,%				use a symbol for spaces in strings
	breakatwhitespace=false,%				sets if automatic breaks should only happen at whitespace
	breaklines=true,%						sets automatic line breaking
	frame=single,%							adds a frame around the code
	captionpos=b,%							where to put the caption
	numbers=left,%							where to put the line-numbers
	numbersep=12pt,%						how far the line-numbers are from the code
	xleftmargin=26.2pt,%					sets left margin for entire box to equal table margin
	framexleftmargin=23pt,%					sets left margin for frame to also surround line numbers
	framexrightmargin=-3.2pt,%				sets left margin for frame to be equal on both sides
}

% Define unformatted language style
\lstdefinelanguage{none}
{
	identifierstyle=
}

\lstdefinelanguage{json}{
	otherkeywords={true, false},
	string=[s]{"}{"},
    literate=
     *{:}{{{\color{black}{:}}}}{1}
      {,}{{{\color{black}{,}}}}{1}
      {\{}{{{\color{black}{\{}}}}{1}
      {\}}{{{\color{black}{\}}}}}{1}
      {[}{{{\color{black}{[}}}}{1}
      {]}{{{\color{black}{]}}}}{1}, 
}

\lstdefinelanguage{CQL}{
	keywords={},
	otherkeywords={
		-,->
	},
	keywords = [2]{MATCH, WHERE, RETURN, COUNT, LIMIT, IN, AS, AND, CREATE, INDEX, CONSTRAINT, FOR, ON, ASSERT, IS, UNIQUE},
	keywordstyle=\color{red},
	keywordstyle=[2]\color{blue},
	comment=[l]{//},
	morecomment=[s]{/*}{*/},
	stringstyle=\color{red}\ttfamily,
	morestring=[b]',
	morestring=[b]"
}


%----------------------------------------------------------
% Define new words with hyphenation
%----------------------------------------------------------

\hyphenation{Graph-daten-bank-system}
\hyphenation{Graph-daten-bank-systeme}
\hyphenation{Graph-daten-bank-systems}
\hyphenation{Graph-daten-bank-systemen}
\hyphenation{be-zieh-ungs-wei-se}
\hyphenation{heran-gezog-en}
\hyphenation{Link-bench}
\hyphenation{Da-ten-bank-mo-dell}
\hyphenation{Da-ten-bank-mo-dells}
\hyphenation{Da-ten-bank-mo-del-le}
\hyphenation{Da-ten-bank-mo-del-len}
\hyphenation{Mo-dell}
\hyphenation{Mo-del-le}
\hyphenation{Mo-del-ls}
\hyphenation{Mo-del-len}
\hyphenation{be-kannt}
\hyphenation{be-kann-te}
\hyphenation{be-kann-ten}
\hyphenation{be-kann-ter}
\hyphenation{be-kann-tes}
\hyphenation{Graph-er-wei-ter-ung}
\hyphenation{Graph-er-wei-ter-ung-en}
\hyphenation{Graph-er-wei-ter-ungs}
\hyphenation{Er-mitt-lung}
\hyphenation{Er-mitt-lung-en}
\hyphenation{über-ge-stülpt}
\hyphenation{gleich-zei-tig}
\hyphenation{Be-schränk-ung}
\hyphenation{Unter-ab-schnitt}

% Manuell zu hyphenaten
% IO-Operation
% IO-Aus\-last\-ung
% Gremlin-Queries